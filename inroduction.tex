\section{背景介绍}

\emph{机器学习的基本框架是什么?数据、模型、训练(优化)、预测?}

机器学习方法三个基本要素:模型、学习准则、优化算法.

在实际任务中使用机器学习模型一般会包含以下几个
步骤(如图1.2所示):
数据预处理:经过数据的预处理, 如去除噪声等. 比如在文本分类中, 去除停用词等.
(2) 特征提取:从原始数据中提取一些有效的特征. 比如在图像分类中, 提取边缘、尺度不变特征变换(Scale Invariant Feature Transform, SIFT)特征等.
(3) 特征转换:对特征进行一定的加工, 比如降维和升维. 很多特征转换方法也都是机器学习方法.
降维包括特征抽取(Feature Extraction)和特征选择(Feature Selection)两种途径. 常用的特征转换方法有主成分分析(Principal Components Analysis, PCA)、线性判别分析(Linear Discriminant Analysis, LDA)等.
(4) 预测:机器学习的核心部分, 学习一个函数并进行预测.



% 你在读该论文时自己也要做好资料搜集工作, 整理联邦学习算法的相关背景(包括为什么要研究联邦学习算法).在这之前, 你需要了解机器学习的基本原理.

\emph{为什么要研究联邦学习?最初提出该概念的目的是什么?结合具体的实例进行说明:隐私性需求、跨数据集、跨平台协作学习}

2016年AlphaGo击败了顶尖的人类围棋玩家,  人类希望人工智能(AI)可以更多的领域发挥作用.
但是现实情况中会遇到相当多问题:
\begin{enumerate}
    \item 隐私性需求:在当今对隐私要求越来越严格的情况下, 如果外部机构使用医院患者, 银行客户的数据, 必须保证数据不泄露; 保险公司渴望应用非保险行业数据提升解决方案能力.
    \item 跨数据集: 在人工智能驱动的产品推荐服务中, 产品销售商拥有产品信息、用户购买数据, 但没有描述用户购买能力和支付习惯的数据.在大多数行业中, 数据以孤岛的形式存在.由于行业竞争、隐私安全和复杂的管理程序, 甚至同一公司不同部门之间的数据集成也面临着巨大的阻力.几乎不可能将分散在全国各地的数据和机构进行整合, 否则成本是难以承受的.
    \item 跨平台: 数据是分散的, 每家应用的数据不一样,  如社交属性数据、电商交易数据、信用数据,  如何进行跨组织间的数据合作, 会有很大的挑战


\end{enumerate}


针对以上问题, 研究人员对此的解决方案有:
\citep{yang2019federated}
\begin{enumerate}
    \item 2016年谷歌提出联邦学习 (the federated learning framework).
          \citep{mcmahan2016communication}
    \item 杨强团队提出一个全面的、安全的联邦学习框架(a comprehensive secure federated learning framework)(包括:横向联邦学习、纵向联邦学习、迁移联邦学习).
          %%\emph{这三类学习机制之间的区别是什么?各有什么应用场景?}
    \item 建议在基于联邦机制的组织之间建立数据网络, 这是一种有效的解决方案, 可以在不损害用户隐私的情况下共享知识.
\end{enumerate}
