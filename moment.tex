 
% \usepackage[paperwidth=21.6cm,paperheight=27.94cm,centering]{geometry}
% \geometry{left=0.5in,right=0.5in,top=2.5cm,bottom=2.5cm}
\documentclass[a4paper]{article}

\usepackage{note}

\title{联邦学习算法研究}
\pagestyle{plain}
\setcounter{page}{1}
\pagenumbering{arabic}

\section{矩}

设X为随机变量,$I$是一个包含$0$的(有限或无限的)开区间,对任意$t\in I$,期望$Ee^{tx}$存在  
则称函数$M_{X}(t)=E(e^{tX})=\int_{-\infty}^{+\infty}e^{tx}dF(x),t \in I$为X的矩母函数 
设X为任意随机变量,称函数$\varphi_{X}(t)=E(e^{itX})=\int_{-\infty}^{+\infty}e^{itx}dF(x)$为X的特征函数 
一个随机变量的矩母函数不一定存在,但是特征函数一定存在。
随机变量与特征函数存在一一对应的关系 随机变量与特征函数存在一一对应的关系

矩生成函数(moment-generating function)又称矩生成函数.

随机变量X的矩生成函数定义为:
$M_X(t)=E(e^{tX}) , t \in \mathbb{R}$
前提是这个期望值存在。

\begin{equation}
    \begin{aligned}
        M_X(t)& = \int_{-\infty}^{+\infty} e^{tx} f(x)dx \\
       & = \int_{-\infty}^{+\infty} (1+tx+\frac{t^2 x^2}{2!}+\dots) \\  
       & =1+tm_1 +\frac{t^2m_2}{2!}+ \dots 
    \end{aligned}
\end{equation}
其中$m_i$是第$i$个矩, $M_X(-t)$是$f(x)$的双边拉普拉斯变化.
 
正态分布$N \sim (\mu,\sigma^2)$的矩母函数

$f_X(x)= \frac{1}{\sqrt{2\pi \sigma}}  e^{-{(x-\mu)}^2}/2\sigma^2 , M_X(s)=e^{(\sigma^2s^2/2)+\mu s}$

\paragraph{马尔可夫不等式}

马尔可夫不等式把概率关联到数学期望,给出了随机变量的分布函数一个宽泛但仍有用的界。 

令 $X$ 为非负随机变量,且假设$ E(X)$ 存在,则对任意的 $a>0 $有
$P\left \{ X \geq a \right \} \leq \frac{E(X)}{a}$

马尔可夫不等式是用来估计尾部事件的概率上界,一个直观的例子是: 

\paragraph{证明:}
$E(X) = \int_{0}^{+\infty}f(x)dx = \int_{0}^{a}xf(x)dx + \int_{a}^{+\infty}xf(x)dx \geq \int_{a}^{+\infty} xf(x)dx \geq a\int_{a}^{+\infty}f(x)dx = aP\left \{ X > a \right \}$

切比雪夫不等式是马尔科夫不等式的特殊情况。
若随机变量 $X$ 的数学期望和方差都存在,分别设为 $E(X)$ 和 $D(X)$,则对任意的 $\varepsilon>0$,有

$P\left \{| X-E(X) | \geq \varepsilon  \right \} \leq \frac{D(X)}{\varepsilon ^{2}}$

通过马尔可夫不等式可证明
$P\left \{| X-E(X) | \geq \varepsilon \right \} = P\left \{[X-E(X)]^{2} \geq \varepsilon^{2}\right \} \leq \frac{E\left \{ [X-E(X)]^{2} \right \}}{\varepsilon ^{2}} = \frac{D(X)}{\varepsilon^{2}}$

切比雪夫不等式没有限定分布的形式,所以应用广泛,但这个界很松。

$\varepsilon$ 代表 X 和期望 $E(X)$ 之间的距离,相差越大,则概率越小,它描述了这样一个事实:事件大多会集中在平均值附近。

\end{document}